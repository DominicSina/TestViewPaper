\documentclass[oneside,a4paper]{book}
\frontmatter
%\pagestyle{headings}
\input{st80.tex} %seems to create a problem with the preamble command needline already defined
\input{preambleTestView.tex}

% A B S T R A C T
% % % % % % % % % % % % % % % % % % % % % % % % % % % % % % % % % %
\chapter*{\centering Abstract}
	\begin{quotation}
		\noindent

The purpose of this bachelor project is to improve the Pharo environment by
making it more unit test friendly. Instead of writing a new system browser we
chose to realize this as a Nautilus plugin since this builds on 
established parts of the Pharo environment. This plugin includes
various functionalities that help with finding and writing unit tests. Furthermore
it provides ways to check if a method is untested, to see all tests that have
been written for a certain method, to add new tests for existing methods and
the ability to view a method and a corresponding test side by side.  Before
explaining the plugin and its functionalities in more detail we will take a
look at various terms surrounding unit testing and analyze the development
environments Eclipse and Pharo. They will be compared to each other in how unit
test friendly they are. We also discuss opportunities
for improvement. The aim of this thesis is to take a closer look at unit testing and in the form
of a Nautilus plugin provide an example of how unit testing can be
facilitated.

	\end{quotation}
\clearpage


% C O N T E N T S
% % % % % % % % % % % % % % % % % % % % % % % % % % % % % % % % % % % % % % % %
\tableofcontents

\mainmatter
%%%%%%%%%%%%%%%%%%%%%%%%%%%%%%%%%%
%%%% Introduction %%%%%%%%%%%%%%%%%%%%%
%%%%%%%%%%%%%%%%%%%%%%%%%%%%%%%%%%
\chapter{Introduction}
	\label{cha:introduction}

Testing has become an important part of software development. While it is true
that ``Program testing can be used to show the presence of bugs, but never to
show their absence!'' \cite{Dijk72b} testing helps to at least partially
validate a program. Tests for example aid verifying newly written code and
detecting if changes broke previously working code (regression errors).

Not only does testing make code more reliable but it also helps to speed up
development by recognizing and preventing regression errors. User studies
indicate that about 10-15\% of the development time is spent waiting for tests
to execute and fixing regression errors. Simply
increasing the frequency with which the tests are executed reduces this wasted
time by 31-81\% \cite{Saff03b}. This means that the additional time it takes to
execute tests more frequently is less than the time saved by fixing regression
errors early.

Since testing provides these benefits it has become a key feature of many
current software engineering paradigms like for example extreme programing (XP)
 \cite{Pres14a}. Together with the other features of these development methods
testing helps to speed up the development and increases software quality. 

Additionally for example in XP the time that is spent
writing a test will be regained. As Wells \cite{Well13a} puts it: ``...during
the life of a project an automated test can save you a hundred times the cost
to create it... ''. This is done by guarding against bugs and regression
errors. Since this guard is in place refactoring and frequent integration
of new code become possible.

While testing clearly has its benefits, especially with fast-approaching
deadlines unit testing is still seen as
optional and too time consuming.  Tests are not executed as often as they
should be which indicates that the use of unit tests is underestimated.  In
their case studies Pressman and Ernst found that simply running the same tests two to five times as
often can reduce the wasted time by 31-82\% \cite{Pres14a}.

Not testing properly will in the long run slow down the development process
since all the previously discussed benefits of testing are missing. So there
might be no time for testing in the future as well.

Breaking out of this circle can require an outside influence \cite{Beck98a}.
Many environments are missing convenient features to facilitate unit testing
and thus repetitively break the developers flow during unit testing. This only
makes breaking out of this circle harder.

In this thesis we attempt to provide such an outside influence for the Pharo
IDE. Our solution is realized as a plugin for Nautilus,
the default system browser in Pharo. The TestView plugin provides
additional functionality such as an easy way to view tests and methods side by side, an
automated test search and a quick way to create new tests.

\chapter{A Closer Look at Testing}
	\label{cha:Testing}

In this chapter we will take a look at different testing practices in order to
specify how these terms are used throughout this thesis. During the next
chapters these terms will then be applied to better describe how current
programming environments support these paradigms and where precisely this
support is lacking. It is important to note that these paradigms are not
mutually exclusive.


\section{Unit Testing}
	\label{sec:Unit Testing}

Unit testing follows the practice of isolating the smallest inseparable parts
of a program and testing them independently of each other. Each of these tests
is done at as low a level as possible. System wide tests are generally not a
part of unit testing.  This means that the scope of each test is very small and
consequently many tests are needed to cover the whole implementation. By making
unit tests one ensures that the tested parts behave as expected. In Smalltalk
these smallest units are often methods since the language heavily relies on
sending messages\footnote{Roughly translates to a method invocation in other
languages} between objects.

To start testing a method an instance of the class under test is created and
brought into a state where a set of preconditions is met.  An example for
preconditions would be the requirement for the object's instance variables to
have certain values. From this initial state in a deterministic environment a
specific outcome can be expected after the method is executed. The outcome is a
success if all of the previously defined postconditions are fulfilled. These
postconditions are checked by using assertions.

To illustrate this let us examine how a unit test for an imaginary
``\emph{Stack}'' class in Pharo might work. Each stack has a maximum
capacity that is set once the stack is created. The stack
provides a method named ``\emph{push: anObject}'' which adds an element on top
of the stack and a method named ``\emph{pop}'' which removes and returns the
element that was added last to the stack.

Now let us examine how to make a unit test for this stack that checks if the
last element that was added is returned by the pop operation and not some other
object. An example of how such a test could be implemented can be found in
\autoref{code:exampleTest}.

The preconditions need to make sure that the stack is initialized with a
capacity bigger than zero. This is done in line \autoref{line:init} of our
example. Then an element is pushed and popped again in line \autoref{line:push}
and line \autoref{line:assert}. The postcondition is checked with the
\emph{assert} that can be found in line \autoref{line:assert}. The popped
element should be the same as the one that was pushed to the stack earlier.

\begin{lstlisting}[caption=Test for a stack's pop method, label=code:exampleTest]
testPopReturnedElement
	| stack |

	stack:=Stack withCapacity: 2. !\label{line:init}!
	stack push: 5. !\label{line:push}!

	self assert: (stack pop=5). !\label{line:assert}!
\end{lstlisting}

For the purpose of this thesis two features of unit testing are important to
keep in mind. The first one is that unit testing takes place at a method level
and the second one is that the narrow scope of unit tests requires many tests
to cover multiple methods. Even a single method is likely to require multiple
corresponding tests since every path in the decision tree of the method should
be covered.

\section{Test-driven Development}
	\label{sec:Test-driven Development}

In Test-driven Development the tests for an implementation are written before
the implementation itself.  Afterwards the implementation is written and
improved until those tests are satisfied.  After this more tests are added to
test additional features. This process is repeated until the implementation is
sufficient.  Below are some basic rules of Test-Driven Development as laid out
by Robert C. Martin \cite{Mart09b}.

\begin{quote}
\textbf{First Law} You may not write production code until you have
written a failing unit test.

\textbf{Second Law} You may not write more of a unit test than is sufficient
to fail, and not compiling is failing.

\textbf{Third Law} You may not write more production code than is sufficient to pass the
currently failing test.
\end{quote}

While it might seem tedious, repeating this careful
planning is the biggest advantage of this approach. Applications developed with
Test-driven Development have to be very thought-out and every requirement has to
be clear since those are required to begin writing the production code.

Test-driven Development bears mentioning because it is one of the major
paradigms concerned with testing and because in the context of this thesis it
has to be treated a bit differently. Namely when trying to facilitate testing
Test-Driven Development can be difficult to properly support. Some useful
information to support the user is only available after a method has been
implemented. This includes the method declaration and the location of this
method. Without these no test name and test location can be derived for new
tests.

\section{Blackbox and Whitebox Testing}
	\label{sec:Blackbox and Whitebox Testing}

Whitebox and blackbox testing refer to whether or not the
implementation of a method is taken into account while writing tests for it.

In whitebox testing the tester is completely aware of the method and its inner
workings when writing a test for it. This means that an adequate amount of unit tests
done with a whitebox approach will cover each important path in the decision tree of a
method.  Especially interesting are the most extreme cases. These are the method
invocations that result in as many executed lines as possible.

Contrary, in blackbox testing, the test writer does not know the
inner workings of the method that is being tested. Ghezzi et al. describe that
during blackbox testing ``test sets are developed and their results
evaluated solely on the basis of the specifications'' \cite{Ghez02a}. This means that
the tests have to be made so that they return the right value for all tested input values.
Since it might be impossible to test all possible input values it is necessary
to choose the critical input values where an error might occur.

%%%%%%%%%%%%%%%%%%%%%%%%%%%%%%%%%%
%%%% The Problem %%%%%%%%%%%%%%%%%%%%%
%%%%%%%%%%%%%%%%%%%%%%%%%%%%%%%%%%
\chapter{The Problem}
	\label{cha:The Problem}

As shown in \autoref{cha:introduction} some current software development
paradigms include unit testing. Still, as important as unit testing is, not
all current development environments are optimized for it.  For example, the
search for unit tests of a specific method is often lacking or not present at
all. In this work we assume that important features for unit test friendly
environments are:

\begin{enumerate}
\item A quick way to see if a method is tested
\item An automated test search
\item The ability to view tests and methods side by side
\item Easy creation of new tests
\end{enumerate}

These criteria are based on the definitions of the various testing paradigms
previously described in \autoref{cha:Testing}.  Namely unit testing can be
supported by letting the developer see if a method is covered by at least one
test, the search for all tests corresponding to a specific method and the
ability to create new tests as fast as possible.  In whitebox testing it is
convenient to see the method under test and the test at the same time without
switching between the two. With this it becomes easier to write tests that
together cover each path in the decision tree of a method.

Since not many environments were created with specifically unit testing in mind,
some of these features are often missing.  Concrete examples of environments
lacking these specific features will be pointed out in \autoref{cha:Related
Work}. To compensate for these missing features the user is required to manually
execute many tasks like navigating back and forth between method and test and
creating new test classes and packages. This repeatedly breaks the programmers'
flow and in effect discourages them from writing tests. In the following
sections we will take a closer look at the tasks involved in different goals
arising from unit testing. This will help us to understand the problem of those
missing functionalities.

\section{Is a Method Tested?}
\label{sec:Is a method tested?}

In unit testing it is important that each method is tested (barring trivial
things like getters and setters). As discussed in \autoref{sec:Unit Testing} a
method with no tests is unsafe and might even have to be tested manually which
is quite slow. Thus a very important feature in unit testing is the ability to
check if a method has at least one test. This has to be as easy to see and as
non intrusive as possible.

If this feature is not implemented users have to switch back and forth between a
class and all its test classes to check manually. If the test classes are not
known to them they have to find them first. Both of those tasks can take a lot
of effort.  Luckily code coverage tools like the Test Runner for the Pharo IDE
or EclEmma for Eclipse also provide this functionality. They allow the user to
run test suites and then see which parts of the implementation were executed.
This will show if a method never got called and thus is untested.

Coverage tools were not designed to check if a method is tested or not. Methods
that are called indirectly by a test suite will still show up as covered.
Developers thus might believe that a method is tested even if it never was
called directly by a test.

\section{Showing All Tests of a Method}
\label{sec:Showing all tests of a method}

Unit testing requires to look up a method and all the tests that have been
written for it.  Just checking if a method is tested at all is not enough, the
developers might want to see if the method is tested sufficiently. To ensure
this multiple test methods are required. Also if at least one test for a method
could be found then the method is not untested. This functionality can thus
easily substitute the one described in \autoref{sec:Is a method tested?}. The
previous ability to see if a method is untested will thus no longer be discussed
separately in this paper.

Similar to showing if a method is tested this helps the user to decide if it is
necessary to write a new test. Furthermore this functionality helps the user to see
if a certain test case has already been created and thus is not needed again. A
developer could also study how a method works by looking at the corresponding
tests.

A lack of this test search function will require great effort on part of the
user to keep track of  test classes and test methods. If the user decides to add
a new test then it is necessary to take a short look at every test to determine
if a similar test has not been written. In this case the tester needs to have
quick access to all tests of this method. A lack of this feature might make unit
testing in larger applications and test suites very tedious and new test writers
will have trouble getting an overview of the existing test suite.

Code coverage tools can not fully substitute an automated test search since it
relies strongly on the relation between tests and method.  To be useful the code
coverage tool would have to save for each method from which test it was called
during the execution of a test suite. Also coverage is not necessarily a good
metric to determine if a test actually tests a method. Nested method calls
might yield a large number of false positives.

\section{Creating New Tests}
\label{sec:Creating new tests}

As discussed previously it should be as simple as possible to check if a method
is not sufficiently tested. If this is the case the user has to create additional
tests.  Especially at the start of a project many new tests will have to be
created. So it is safe to say that another reoccurring user goal in unit testing
is the creation of new tests for insufficiently tested methods.

While creating a new test the developer first has to decide to which package and
class the new test will be added and how the test will be called. In many
cases the location of this new test has already been decided on since other
tests corresponding to the same method are already there. In this case a quick
option to specify an existing test class should be offered.

If there are no test classes and packages or if the existing ones do not fit the
new test then the user has to create a new test class. In this case the test
package, test class and test names are often derived from the original package,
class and method names. A slight drawback of these default name derivations is
that the method that is being tested has to be defined previously.
The programming environment can
not always make such name derivations especially during Test-driven
Development.

As shown in the last two paragraphs it does not matter if the test class and
package already exist or if they need to be added. Creating and placing new
tests can be facilitated either by letting the user quickly specify existing
test classes or by making the creation of new test classes easier. Another helpful
feature is the ability to let the user create these packages, classes and tests
together instead of individually.

If the default names are missing the developer has to enter similarly structured
names for packages, classes and tests over and over. In case that test packages,
classes and methods can not be created together the user will have to interact
with the environment multiple times to start the creation of each of these.
While it helps to have these aids a lack of them seems not as bad as a lack of
the test search functionality described previously in \autoref{sec:Showing all
tests of a method}.

\section{Methods and Tests Side by Side}
\label{sec:Methods and Tests Side by Side}

In whitebox testing it is required to know a method's inner workings. It is
expected that the programmer who writes the test is trying to test every line of
code. Thus it is important to provide the user with information on how the
method works. Possibly the simplest way is to let the users see the whole source
code of the method in question at the same time as they are writing a test for
it.

In blackbox testing this feature can be counterproductive by distracting the
users from the method specifications. The additional information can thus spoil
blackbox testing and even if completely ignored still takes up space on the
screen. Blackbox testing thus requires that this feature can be turned off.

If during whitebox testing the environment does not provide information about a
method the users have to gather it themselves.  This can be done by taking
notes, switching back and forth between method and tests or opening an
additional window to show both at the same time.

This can be very tedious and take a lot of time. Making it unnecessary to switch
between gathering information about a method and writing tests for it can save
valuable time and effort.

%%%%%%%%%%%%%%%%%%%%%%%%%%%%%%%%%%
%%%% Related work %%%%%%%%%%%%%%%%%%%%%
%%%%%%%%%%%%%%%%%%%%%%%%%%%%%%%%%%
\chapter{Related Work}
	\label{cha:Related Work}

\begin{figure}[b!]
	\centering
	\includegraphics[scale=0.4]{screenshots/eclipseIntro3.png}
	\caption{Eclipse's user interface}
	\label{fig:Eclipse's user interface}
\end{figure}

In this section we describe current programming environments, namely Eclipse and
Pharo with respect to how unit test friendly they are. Special focus will be
placed on the features described in \autoref{cha:The Problem}. These features
are: viewing test and method side by side, search all tests for a specified
method and facilitating the creation of new tests. The two environments are
compared and possible places for improvement are discussed.

\section{Unit Testing in Eclipse}
\label{seg:Unit Testing in Eclipse}

We will start by taking a closer look at Eclipse since it is widely used and
many parallels to other programming environments (\eg Visual Studio) can be
drawn.

In \autoref{fig:Eclipse's user interface} you see the Java project JHotDraw 7.6\footnote{http://www.randelshofer.ch/oop/jhotdraw/} opened in Eclipse 4.5.1
with multiple packages and classes. The packages and classes
can be seen on the left side in the ``Package Explorer'' pane.

\subsection{Methods and Tests Side by Side}

We start with a discussion on how Eclipse enables users to see a
method and corresponding tests at the same time.

One option is to open both classes through the Package Explorer. The opened
classes will be shown in the middle of the screen and a new tab on top of the
code editor will appear.  In \autoref{fig:Eclipse's user interface} there are
two tabs ``\emph{BezierPath.java}''  and ``\emph{BezierPathTest.java}''.  The
``\emph{BezierPath.java}''  tab is currently selected and thus the content of
``\emph{BezierPath.java}'' is displayed in the code editor.  Unless you close
these tabs all the classes you opened will be quickly available through their
tabs.

A big advantage of this system is that users can create favorites by not closing
important tabs. Through this they can switch back and forth between methods and
tests. A slight drawback is that the user has to close unneeded tabs from time
to time to more quickly find the currently important ones. In most cases the use of
these tabs will be smooth enough to allow the programmer to switch uninterrupted
between classes and their tests.

If for some reason this does not suffice then it is also possible to split the
code editor multiple times either horizontally or vertically like in
\autoref{fig:eclipseSplitscreen}. Since each code panel is smaller than the
original one this splitting can only be done a limited number of times. After
this the panels get too small to be useful.

As shown, Eclipse provides multiple ways to make the tested method easily
accessible during testing.

\begin{figure}[h]
	\centering
	\includegraphics[scale=0.4]{screenshots/eclipseSplitscreen3.png}
	\caption{Split code panel}
	\label{fig:eclipseSplitscreen}
\end{figure}

\subsection{Test Search}

Another feature we discussed in the previous chapter was the ability to find all
tests of a certain method. Eclipse provides a good option to search for tests.
By selecting the ``Search'' from the top menu bar followed by ``Referring Tests...''
it is possible to find all tests that reference the currently selected method.

Sadly this feature is very badly documented. The requirements for a test to
be associated with a certain method seem to include that the test class
extends ``TestCase'' and that the test name starts with ``test''. The ``@Test''
annotations are ignored by this function. It seems like this test search does
not yet support JUnit 4 standards.

\subsection{Creating New Tests}

The next feature in Eclipse that we now discuss is how the creation of new tests
is facilitated. Eclipse provides an option for adding a new test class by right
clicking on the class that should be tested, selecting ``New'' from the list that
is shown and then clicking ``JUnit Test Case''.

In the newly opened wizard a default name is already created and the selected
class is put into the ``Class under test'' field, as shown in
\autoref{fig:eclipseWizard1}. As default package the package of the class under
test is used but it can be changed. By clicking on the ``Next'' button it is
possible to select methods of the class under test and create test stubs for
every selected method, see \autoref{fig:eclipseWizard2}.

\begin{figure}[b!]
  \centering
  \begin{minipage}[b]{0.495\textwidth}
    \includegraphics[width=\textwidth]{screenshots/eclipseWizard1_3.png}
    \caption{Test class wizard}
	\label{fig:eclipseWizard1}
  \end{minipage}
  \hfill
  \begin{minipage}[b]{0.495\textwidth}
    \includegraphics[width=\textwidth]{screenshots/eclipseWizard2_3.png}
    \caption{Select methods to test}
	\label{fig:eclipseWizard2}
  \end{minipage}
\end{figure}

On one hand, this wizard is very good for creating new test classes but on the
other hand Eclipse helps neither to add new test methods in an already existing
test class nor to create more than one test per method. During Test-Driven
Development its use is also limited, since it can only add tests to existing
methods.

\section{Unit Testing in Pharo}

Now let us take a look how the Pharo IDE and its system browser Nautilus support
unit testing and how they compare to Eclipse. In this context it seems worth
mentioning that the Pharo IDE is quite different from Eclipse in that the
placement of its visual elements is less rigid. The users are able to customize
the appearance of the environment very quickly and adapt it to their wishes. On
the other hand Pharo is less wide spread and not as well maintained in many ways
as Eclipse.

Like in \autoref{seg:Unit Testing in Eclipse} the discussed features are viewing
test and method side by side, searching all tests for a specified method and
facilitating the creation of new tests.

\subsection{Methods and Tests Side by Side}
\label{subsec:Nautilus side by side}

As with Eclipse the first functionality we will look at is the ability to view
tests and methods side by side. The Pharo IDE provides a very customizable
environment. It invites users to arrange all windows that are created in a way
that seems comfortable. It is possible to create multiple Nautilus windows and
arrange them so that one shows the method and the other the test for this
method. Through this a very similar effect to Eclipse's split code editor is
achieved.

The free placement of those Nautilus windows gives the user more freedom to
customize the environment but several visual elements will be duplicated which
reduces the available space on the screen to arrange these. An example for such
a duplicated element is the list of packages and classes that takes up the top
half of each Nautilus window like for example in \autoref{fig:Locking}.

A possibility that does not have this drawback is to lock a method or class and
then select others inside the same Nautilus window. It is possible to select
which of the locked elements is shown in the code panel by using the ``All'',
``Current'' and number buttons shown at the bottom of \autoref{fig:Locking}.
Through these buttons the users can switch between seeing all locked methods and
classes at the same time or each individually. This is very similar to the tabs
of opened classes that Eclipse has. Although with many locked classes and
methods it is hard to remember which number belongs to what.

 \begin{figure}[h!]
	\centering
	\includegraphics[scale=0.4]{screenshots/pharoLockedMethods2.png}
	\caption{Nautilus window with active locks}
	\label{fig:Locking}
\end{figure}

Apart from this Nautilus also has the History Navigator which allows fast access
to all recently viewed methods and classes. The History Navigator can be seen in
the middle right in \autoref{fig:Locking}. It can be used to switch between
method and test in a way that is comparable to Eclipse's tabs of recently opened
classes. The drawback here is that only a certain number of those recently
accessed classes and methods is stored and if the user looks at different
methods and classes it is quickly necessary to reopen the previous tests and
methods to put them back in the History Navigator.

To conclude in the Pharo IDE are various things a user can do to view multiple
pieces of code at the same time. Compared to Eclipse there are more ways to see
things side by side or to switch between them but each of these ways has a
slight drawback.

\subsection{Test Search}
\label{subsec:Test Search}

The next functionality that will be discussed is how Pharo finds existing tests
for a specified method. Like Eclipse Nautilus has a test search implemented.
Like with Eclipse's test search only tests with very specific
names are found. Additionally they have to be placed in classes with an
equally specific name. Methods and classes with a different name than what
follows are not considered corresponding test classes and methods.

Namely, test classes and test methods have to contain the full name of the class
or method under test. Additionally the name of the test class has to have the
suffix ``Test'' and the name of the test method has to have the prefix ``test''.
Also tests have no parameters and thus lack all colons in their name.  If
anything more is part of the test name then it will not be found by Nautilus's
test search. The test classes also have to inherit from
\emph{TestCase} in order to be found by the test search. Examples of
tests corresponding to certain methods can be found in
\autoref{tab:correspondingMethods}.

\begin{table}[h!]
	\centering
	\begin{tabular}{| c | c || c | c |}
     	\hline
     	\emph{Class name} & \emph{Method name} &  \emph{Test class name} & \emph{Test name}\\ \hline
	AClassName & aMethodName & AClassNameTest & testAMethodName\\
	Nautilus & selectedClass & NautilusTest & testSelectedClass \\
	RxMatcher & matches: & RxMatcherTest & testMatches \\
	Stack & push: & StackTest &  testPush \\
	ProtoObject & ifNil:ifNotNil: & ProtoObjectTest & testIfNilIfNotNil \\
     	\hline
   	\end{tabular}
	\caption {Methods and corresponding tests found by Nautilus}
	\label {tab:correspondingMethods}
\end{table}

Both the limitations on class names and method names are quite restrictive.
From a test class name the class under test can be quite reliably inferred. On
the other hand the method name and the corresponding test name for it are less
directly related \cite{Mars05a}. These restrictive name criteria also lead to the
problem that at most only one test class or test method will be found for each class or method since
others with a slightly different name will not fulfill these naming criteria.

This test search functionality is not directly available to the developers. It is
used by Nautilus  in the file hierarchy to add a button to the left of each
method or class where corresponding tests have been found. This button can be pressed to execute the
corresponding test method or respectively all tests contained in the corresponding test class and show if these corresponding tests were successful or not.

The test search is better in Eclipse since it is not only less restrictive with
its name-based criterion but also checks if the test calls the method under
test. Additionally Eclipse's test search is able to find more than one test for a
method. On the other hand in Nautilus the test search is better integrated in
the environment. The additional button to execute a corresponding test
conveniently allows running the test without navigating to it.

\subsection{Creating New Tests}

Right clicking on a method in Nautilus gives the ``Generate test''  option.  The
new test will automatically be added to a certain test class in a specific
package (both of which will be created if needed). More precisely the test
package is named like the package under test with the suffix ``-Tests''. The test
class and method will be named like \autoref{tab:correspondingMethods} shows.
All the tests created through this will thus be found by Nautilus's test search.  
``Generate test and jump'' works very similar but also selects and
shows the newly created test so that the user can start implementing.

Using these predefined package and class names is faster compared to Eclipse's test wizard since the 
user does not need to confirm them. Another advantage is that tests created in this manner automatically
confirm to the naming standards imposed by Nautilus. Sadly this robs the user of
the ability to specify different package and class names. Also only one test per
method can be created in this manner. All additional tests for the same method
will simply overwrite the previous test since they will have exactly the same
name.

Eclipse's and Nautilus's way of adding new tests are very comparable. A case
could be made for both versions of this feature. Eclipse's version is more
customizable but only supports the user when creating a new test class while
Nautilus's version is faster and supports the user with each new test but is
less flexible. An easy way to create multiple tests for one method is
provided by neither Nautilus nor Eclipse.

%%%%%%%%%%%%%%%%%%%%%%%%%%%%%%%%%%
%%%% The TestView Plugin %%%%%%%%%%%%%%%%%%%%%
%%%%%%%%%%%%%%%%%%%%%%%%%%%%%%%%%%
\chapter {The TestView Plugin}
	\label{cha:The TestView Plugin}

Having discussed some important features of a unit test friendly environment in
\autoref{cha:The Problem} and where their implementation is lacking in
\autoref{cha:Related Work} we will now present an attempt to address this
problem: the TestView Plugin.

Since the lack of unit test supporting features can not be corrected for all
environments at once we chose to extend Nautilus's functionality. The reasoning
behind this was that it is much easier to implement a plugin for Nautilus than
for Eclipse. The inability to view tests and methods side by side
in a single Nautilus window was perceived by us as the gravest of these issues.

In this chapter we will comment on how the TestView Plugin performs regarding
viewing test and method side by side, finding tests for certain methods and
creating new tests. Comparisons to Eclipse and the Pharo IDE will be drawn.

\section{Methods and Tests Side by Side}

As stated in \autoref{subsec:Nautilus side by side} one of the drawbacks
concerning unit testing with Nautilus is that it is hard to view tests and
methods at the same time. Whitebox testing becomes cumbersome through this.
Seeing both method and test together should be quick and convenient. It should
also not introduce too much redundant user interface elements that might clutter
up the environment.

Following this, the TestView Plugin allows splitting the code editor panel of a
Nautilus window vertically into two parts. The result of this can be seen in
\autoref{fig:Nautilus window with two code editors}. With this approach the need
to open two Nautilus windows is eliminated. The list of classes and packages for example
will not be displayed a second time and less screen space is occupied.

\begin{figure}[h!]
	\centering
	\includegraphics[scale=0.4]{screenshots/toggledOn2.png}
	\caption{Nautilus window with installed TestView Plugin}
	\label{fig:Nautilus window with two code editors}
\end{figure}

It also becomes unnecessary to use Nautilus's locking feature or History
Navigator to be able to switch quickly between methods and tests. Thus it also is
not required anymore to manage the locked or recently viewed classes and methods
in order to quickly navigate back and forth.

Unlike in Eclipse the user has no absolute control over this additional code
editor. Only methods of interest are shown there and the user does not have to
manage what should be displayed. This is possible due to the fact that some
assumptions can be made about what code the user wants to see if this code
editor is only used during unit testing. The left code editor will be displaying
the method selected in the file hierarchy and the right code editor will show
the corresponding tests.

Which test is currently shown can be changed through the drop down list that can be
seen in \autoref{fig:Nautilus window with two code editors} just below of the
Nautilus History Navigator.

This approach combines the convenience of Eclipse's ability to show code side
by side with limited but more focused content in the right code editor. Through this
additional code editor the user always has quick access to a method's inner workings.
This is especially useful during whitebox testing. Since it is not possible to view multiple methods in one
Nautilus window at the same time this is a notable improvement.

\section{Test Search}
\label{sec:TVtestsearch}
Since the additional code editor introduced by the plugin is only supposed to shows tests
 it became necessary to use a test search. Without this it would be
impossible to show a relevant selection of corresponding tests. As discussed in \autoref{subsec:Test
Search} the test search that Nautilus provides is very restrictive and only
finds at most one test per method. In order to find better results the TestView
Plugin uses its own test search implementation.

In the following subsections we will provide a detailed explanation how
corresponding tests for a certain method are found. The plugin uses a
hierarchical search with two stages. First the TestView Plugin searches all test
classes of the currently selected class and then inside of these test
classes all test methods of the currently selected method.

Our test search expects test classes to inherit from ``\emph{TestCase}'' and to have a similar name as 
the class that they are testing. The user is able to manually add or remove test classes 
from the search results. A test method has to either have a similar name as the method under test 
or directly call a method with the same name as the method under test. 

How exactly the test class search and test method search work will be discussed in \autoref{subsec:CorrTestClasses} and \autoref{subsec:CorrTestMethods}.

This search is performed every time a method is selected through Nautilus. This makes
it possible that the additional code panel introduced by the plugin always shows
a relevant selection.

\subsection{Corresponding Test Classes}
\label{subsec:CorrTestClasses}

To qualify as a test class a class has to inherit from ``\emph{TestCase}''. This requirement
has to be fulfilled in Nautilus' test search as well.

After this a name-based search is performed on the remaining classes.
A name-based search works quite well to find test classes corresponding to a
certain class \cite{Mars05a}. A name-based criterion is used in Nautilus' test search
and something similar is also part of the test class search in the TestView Plugin.

As discussed in \autoref{subsec:Test Search} Nautilus' naming criteria result in at most one
corresponding test class being found. The naming criterion that the TVPlugin uses is less
restrictive.

The name-based criterion of the TVPlugin requires the name of the class in question to contain
the full name of the selected class in addition to ``test''. The name of the 
selected class and ``test'' have to be contained separately in the name of a potential test class to
fulfill this naming criterion. 
Unlike Nautilus' naming criterion this substring search is not case sensitive and it does not matter 
which of those substrings comes first or if the name contains additional characters.

In \autoref{tab:TVTestClassNames} examples of test class names that fulfill and that do not fulfill
the naming criterion of the TestView Plugin are shown.

\begin{table}[h!]
	\centering
	\begin{tabular}{| c | c | c |}
     	\hline
     	\emph{Original Class Name} & \emph{Possible Class Names} &  \emph{Not Test Classes}\\ \hline
	String & StringTest, stringtest, TestString & String, Test, StrinTgEST \\
     	Contest & ContestTest, contesttest & Contest \\
     	\hline
   	\end{tabular}
	\caption {TestView name criterion for test classes}
	\label {tab:TVTestClassNames}
\end{table}

To closer illustrate this substring search let us examine why a test class named
``Contest'' is not considered a corresponding test class for a class that is also 
named ``Contest''. To confirm to the naming criterion used by the TVPlugin the test 
class name ``Contest'' would have to contain the full name of the class it is 
supposed to test (in this example ``Contest'') as well as ``test''. 
While ``Contest'' indeed contains both of those substrings they are overlapping.
The only occurrence of ``test'' is inside of ``Contest''. So the plugin does not consider
``Contest'' to be a test class for a class also named ``Contest''. To 
make this test class confirm to the naming criterion it could be renamed to
``ContestTest''.

Since the test method search builds on the results of the test class search it
became important to make the test class search works works as good as possible.

Our solution to this is to let the users customize the search results. If the
search for corresponding test classes is inadequate the developers can use the
``Link Class'' and ``Unlink Class'' buttons provided by the TestView Plugin to
add and remove test classes from the results.

Classes that have been unlinked will no longer be searched for test methods
while linked classes will be searched even if they do not inherit from
``\emph{TestCase}'' or do not fulfill the naming criterion.

The plugin keeps track of the linked and unlinked classes in two
\emph{Dictionaries}. When a class is linked or unlinked the currently selected
class is used as a key to store the specified class. Since the currently
selected class is used as a key each linking or unlinking of a class only counts
for the currently selected class.

\subsection{Corresponding Test Methods}
\label{subsec:CorrTestMethods}

After some possible test classes have been identified using the criteria
described in \autoref{subsec:CorrTestClasses} the resulting classes will be
searched for corresponding test methods of the currently selected method.

Similar to the name criterion for test classes described in
\autoref{subsec:CorrTestClasses} the test methods have to contain the full name
of the selected method (without colons) as well as the string ``test''. Examples
can be found in \autoref{tab:testNameCrit}.

\begin{table}[h!]
	\centering
	\begin{tabular}{| c | c | c |}
     	\hline
     	\emph{Original method name} & \emph{Test names} & \emph{Not test names} \\ \hline
    	 addDays: & testAddDays & addDays, testAddDay \\
    	 asFloat & testFractionAsFloat, testFractionAsFloat2 & testFloatAsFraction \\
	print24:on: & testPrint24On, testPrint24OnWithPM & testPrint24withNanos \\
	 attest & attestTest, testAttest & attest \\
	\hline
   	\end{tabular}
	\caption {Test naming criterion}
	\label {tab:testNameCrit}
\end{table}

The substring search works as described in \autoref{subsec:CorrTestClasses}.
A test method name has to contain both the full name of the selected method
as well as ``test''. Both of these substrings have to be contained separately \ie 
they can not overlap. If both substring checks are successful the method is a 
test for the selected method.

The second criterion besides the name criterion is that the test contains a
method invocation with the same selector as the selected method. In other words
the test has to directly call a method with the same name as the selected method to fulfill this criterion.
A method is considered a test if it either fulfills the naming criterion or directly calls a method with the same name
as the method that it is supposed to test.

How many of those criteria are met is used to order the found tests in the
drop down list that can be seen in
\autoref{fig:Nautilus window with two code editors} right below of the Nautilus
History Navigator. Specifically each found test method has an associated rating.
The rating is increased by one if the method uses the selected method and by two
if the naming criterion is fulfilled. Test methods with a higher rating will be
displayed first in the results.

This new test method search should yield better results than Nautilus' search since it
allows multiple test methods for one method. It works very
similar to the test search provided by Eclipse and like Nautilus' search is
used directly by the IDE and does not have to be called manually.

\section{Creating New Tests}

Another purpose of the TVPlugin is to make the creation of new tests as quick as
possible. This is done by providing the user with default names as much as
possible and making the creation of new test packages, classes and methods as
easy as possible.

Every time a developer starts to write a new unit test they have to determine to
which test class the new test belongs. We can split the creation of new tests
into two basic use cases: adding the new test in a new test class or adding the
new test in a test class where tests corresponding to the selected method
already exist.

\subsection{Adding a New Test Class}
\label{subsec:AddToNewTestClass}
First let us talk about what can be done to facilitate the creation of a new
test that does not yet have a fitting test class.  The easiest way to do this is
to let the users write the new test and later determine where this test will be
placed. With this the user is encouraged to immediately start writing a test as
soon as a method is created. The new test class option is activated by selecting
the topmost element ``new Test'' from the drop down list of found tests as shown in
\autoref{fig:newTest}.

\begin{figure}[h!]
	\centering
	\includegraphics[scale=0.4]{screenshots/dropListNewTest.png}
	\caption{Expanded drop down list with new test option}
	\label{fig:newTest}
\end{figure}

When the test method is saved the plugin will ask the user for the name of the
test class and in which package this class should be placed. Default values,
based on the class name and the package name of the method under test, are
provided. This might already be sufficient since test class names and test
package names can often be derived from the original class and package names
 \cite{Mars05a}.

To create the default name for a new test package ``-Test'' is added to the name
of the package under test. Similarly the default test class name is derived from
the name of the class under test with the suffix ``Test''.

For example if the method under test is contained in a class called ``Queue''
within a package ``Collections'' then the proposed names for the test class and
the test package will be ``QueueTest'' and ``Collections-Tests''
respectively. The new test class will also automatically subclass
TestCase.

These names and the inheritance from TestCase is also what Nautilus's test
search expects and thus this does not break existing conventions. Even though
default names are provided the user can still ignore them and specify different
ones.

\subsection{Adding a Test to an Existing Test Class}

The second use case is that a developer wants to add a test for a method that
already has tests. In this case the corresponding test classes should have
already been found by the test search described in \autoref{sec:TVtestsearch}.
The users just have to select any found test
that is in the test class where the new test should be added. No test package
has to be specified since only when a new test class is made the package might
also be a new one.

When the user saves a test while having selected the ``new Test'' option described in
\autoref{subsec:AddToNewTestClass}, the plugin will always ask the user to give names
for the test class and the test package. When the user has any other item
selected in this drop down list then the test is saved in the test class that was selected.

Now let us take a look at how this functionality compares to Eclipse and
Nautilus. Improvements compared to Eclipse are that the user can add new tests
in an already existing test class. Eclipse has an advantage when
creating a new test class and filling it with multiple new tests. Eclipse
supports the user only at the start of writing a new test class while the
TVPlugin keeps supporting the user during the addition of any new test.

In Nautilus the user can extremely quickly create new tests in fixed test
packages and test classes but only one test per method can be created that way.
If the ``Generate test'' or the ``Generate test and jump'' option is pressed
multiple times then the previous test is overwritten. The TVPlugin helps the
programmers to add multiple tests to the same method.

%%%%%%%%%%%%%%%%%%%%%%%%%%%%%%%%%%
%%%% The Validation %%%%%%%%%%%%%%%%%%%%%
%%%%%%%%%%%%%%%%%%%%%%%%%%%%%%%%%%
\chapter {The Validation}
	\label{cha:The Validation}
%intro(aim, evaluation methods, )

To see if the TestView plugin fulfills our initial aim to encourage unit testing
we conducted a small scale controlled experiment coupled with a questionnaire.
Our most important research question was to see if the TestView plugin increases
test coverage by reminding the users to write tests. Other research questions
were if the plugin reduces the time, clicks and keystrokes necessary to
implement some simple classes and methods.

\section{The Setup}
%the participants
We evaluated the plugin with four PhD students from the Software
Composition Group of the University of Bern. The only requirements for the
participants were that they know both the Pharo IDE and programming language.

%the evaluation
The evaluation was printed on multiple sheets in order to prevent later
questions and tasks from influencing the participants. The evaluation started
with two simple multiple choice questions surrounding test-driven development
and unit testing. This was done to be able to group the participants according
to how comfortable they were with testing. These questions were:

\begin{quote}
\begin{enumerate}
\item \label{itm:q1} Do you know what Test-driven Development is?


Yes		No
\item \label{itm:q2}How often do you write unit tests for your code?


Never		Rarely		Often		Always
\end{enumerate}
\end{quote}

The next part of the evaluation was the controlled experiment. For this we
selected two problems in advance. The participants were
asked to write in the Pharo IDE a solution to these problems. One problem was
the ``Even Fibonacci
numbers'' problem\footnote{https://projecteuler.net/problem=2} and the other
problem was the ``Sum Square
Difference'' problem\footnote{https://projecteuler.net/problem=6}.

We paraphrased these problems taken from Project Euler to ensure that both
tasks would require the same number of methods to complete:

\begin{quote}
\label{prob:fib}
Task1

Implement the ``Even Fibonacci numbers'' problem.

(Hint the Fibonacci numbers are: 1,2,3,5,8,13,21,34,55,89,...)

a) Write a class with a method that generates all Fibonacci numbers below a
certain number ``collectFibBelow: aNumber''.

b) Additionally the class should have a method to sum up all even numbers in a
collection.

c) Write a method that combines a) and b) and sums up the even Fibonacci
numbers below a certain number.

\end{quote}

\begin{quote}
\label{prob:sumDiff}
Task2

Implement the  ``Sum Square Difference'' Problem.

a) Write a method ``sumSquaresOf: aCollection'' that receives a collection of
numbers, squares each number and sums up the squares.

b) Write a second method ``squareSumOf: aCollection'' that receives a collection
of numbers, sums them up and returns the square of the sum.

c) Write a third method ``sumSquareDiff: aCollection'' that calculates the
difference between ``squareSumOf: aCollection'' and ``sumSquaresOf: aCollection''.

\end{quote}

The participants were required to use a prepared laptop with two Pharo images.
One image had the TV Plugin installed and the other did not. In each image one
of the tasks had to be implemented.

To keep track of the time it took the participants to complete the tasks
and the user interactions with the IDE we used DFlow\footnote{http://dflow.inf.usi.ch/experiment.html}.
DFlow is a tool written for the Pharo IDE that tracks user interactions like mouse movement, key presses
and various windows opened inside of Pharo \cite{Mine2016a}. We modified DFlow slightly to correctly keep track of mouseclicks and made it log the gathered data localy instead of sending it to a server.

In order for the sequence of the images and the difficulty of
the problems not to influence the evaluation we used
counter balancing on both of these factors.  This means that the first two
people started both with the ``Even Fibonacci numbers'' Problem while the
last two people ended with this problem. Every participant with an odd number
started without the TVPlugin while those with an even number started the first
task with it and then completed their second task without it.

The evaluation ended with an open question to gather opinions about the
usefulness of the plugin:

\begin{quote}
\begin{enumerate}
\setcounter{enumi}{2}
\item \label{itm:q3}Did the TestView plugin help in your opinion?
\end{enumerate}
\end{quote}

To measure if the plugin encourages the users to write tests we used the test
coverage percentage. We neither told the participants explicitly to write tests
nor that the plugin was there. With this we wanted to see if simply by being
there the plugin would remind users to write tests. Only if the participants
decided to write tests we told them to use the plugin if they did not notice
it.

To quantify the participants' performance  we took additional measurements.
With these the participants that used the plugin and those that did not should
be compared. This should give us an indication if the plugin facilitates
testing or not.

\section{The Results}

As expected everyone answered question \ref{itm:q1} with yes. For question
\ref{itm:q2} two participants stated that they write unit
tests rarely and two stated that they write unit tests often.

The results of the experiments can be seen in \autoref{tab:eval1} and
\autoref{tab:eval2}. The results seem pretty unexpected which might be due to
the small number of participants. All measurements were taken with DFlow. The
tracked values are:

\begin{enumerate}
		\item time to complete the task in seconds
		\item number of pressed keyboard keys
		\item number of mouse clicks
		\item number of opened windows
		\item test coverage percentage
\end{enumerate}

\begin{table}[h]
	\centering
	\begin{tabular}{| c | c | c | c | c | c | c | c |}
     	\hline
     	Participant&Problem&First Task&Time&Keys pressed&Clicks&Windows&Coverage \\ \hline
    	1&	SumDiff	&No		&581		&540		&65	&12	&0\\
	2&	Fibonacci	&Yes		& 1945	&1550 	& 542	& 24	& 100\\
	3&	Fibonacci	&No		&1089		&810		&287	&17	&0\\
	4 &	SumDiff	&Yes		&516		&486		& 91	& 5	& 0\\  \hline
	Average:&		&		&1033		&847		&246 &15 	&25\\
	\hline
   	\end{tabular}
	\caption {Results with the TV Plugin}
	\label {tab:eval1}
\end{table}

\begin{table}[h]
	\centering
	\begin{tabular}{| c | c | c | c | c | c | c | c |}
     	\hline
     	Participant&Problem&First Task&Time&Keys pressed&Clicks&Windows&Coverage \\ \hline
    	1 &	Fibonacci	&Yes	&702	& 687		& 89	&13	& 0\\
	2&	SumDiff	&No	&919	&739		&170	&9	&100\\
	3 &	SumDiff	&Yes	&433	& 261		&107	&4	& 0\\
	4&	Fibonacci	&No	&1023	&1178		&108	&8	&0\\ \hline
	Average&		&	&769	&716		&119	&8.5 	&25\\
	\hline
   	\end{tabular}
	\caption {Results without the TV Plugin}
	\label {tab:eval2}
\end{table}

The averages of the experiments with the TV Plugin seem to be slightly worse
than those without. This is very surprising since except participant number two
 nobody used or noticed the plugin. Most likely the small number of participants made the differences
between them the decisive factor.

The participants, with the exception of participant number, two did not write unit tests or make
use of the TestView plugin. The one participant that used the plugin has a test
coverage of 100\% with or without the plugin. Also Person two was faster,
pressed less keys or mouse buttons and opened less windows
without the plugin. This difference is most likely due to the varying difficulty
of the problems or due to the participant not being familiar with the
plugin. Everybody seemed to have more difficulties with the
\hyperref[prob:fib]{Fibonacci problem}.

Is is interesting to note that while only one person wrote
unit tests everyone was testing if parts of their code worked as expected in
the Pharo Playground.  Testing definitively is a necessity but sometimes
developers seem to prefer to do it manually. By making unit testing easier a
lot of those manual tests could hopefully become unit tests. That everybody
answered positive to question \ref{itm:q3} underlines the need for additional
unit testing tools and an environment that properly supports unit testing.

The participants also gave general feedback to the plugin. The two most
mentioned points were that the plugin should stand out more and that there
should be a quick option to run all the found tests.

If making the plugin stand out more is really necessary is debatable since this
was only wished by people who did not see the plugin. While this certainly
would help users to discover the plugin, other users that already know about it
could possibly be annoyed by it standing out too much.

A quick option to run all the found tests would certainly be a useful addition
to the plugin. Luckily the Pharo IDE already provides the
TestRunner, a tool to execute tests and keep track of the test coverage of those tests.

Nautilus itself allows the execution of tests by pressing a button next to each method or class
that has corresponding tests. Sadly it is quite possible that not all tests get executed through
this since as described in \autoref{subsec:Test Search} only one corresponding test per method
or class is found by Nautilus.
So while implementing a functionality to run tests through the TestView Plugin might be an option,
enhancing the existing tools would keep the number of required tools low.

Other possible points for improvement that some participants mentioned were to include
the \emph{setUp} method in the drop down list of found tests and to enhance the
template for new tests so that more \emph{assert} variations would be used.

%%%%%%%%%%%%%%%%%%%%%%%%%%%%%%%%%%
%%%% Conclusion and Future Work %%%%%%%%%%%%%%%%%%%%%
%%%%%%%%%%%%%%%%%%%%%%%%%%%%%%%%%%
\chapter {Conclusion and Future Work}
	\label{cha:Conclusion and Future Work}

Unit testing is an important part of current software development paradigms
like Extreme Programming. Sadly in many software development environments unit
testing is not as well supported as it could be. This lack of support can break
the developers' flow and discourage them from testing.

In \autoref{cha:The Problem} we examined Eclipse and the Pharo IDE with its
system browser Nautilus to see where exactly unit testing could be facilitated.
Features that are lacking include: the ability to conveniently view methods and
their tests side by side,  an automated test search and an option to easily
create new tests.

In an attempt to fix these problems concerning unit testing for the Pharo IDE
we made a Nautilus plugin called TestView Plugin or TVPlugin.

With the TVPlugin it becomes possible to horizontally split the code editor of
a Nautilus window. This second code editor will only be used for tests so
certain optimizations can be made. Namely the environment can derive which
tests the user wants to see in this second editor from whats displayed in the
first code editor.

To be able to always show relevant tests the TVPlugin makes use of its own test
search. Improvements compared to Nautilus' test search include less restrictive
criteria for a method to qualify as a corresponding test. Additionally one
method can now have multiple corresponding tests. See
\autoref{sec:TVtestsearch} for a more detailed explanation of how the search
works and what exactly the new criteria are.

To facilitate the creation of new tests the plugin provides default names and a
template for each new test. Both adding a test in a new test class or an
existing one is supported.

To evaluate if the TVPlugin encourages and facilitates testing we conducted a
controlled experiment with a short questionnaire. The results although were not
very decisive. While all participants agreed that the TVPlugin could help writing tests the
recorded measurements showed no clear improvement. This might be due to the
low number of participants and the differences in their programming styles.
Another user study would be needed to show the effectiveness of the plugin.

The participants of the user study suggested to implement an option to run all
tests found by the plugin as well as to enhance the test search so that the
\emph{setUp} methods also would be found. More debatable but wished for by many
participants is to make the plugin stand our more visually.

Another possible improvement would be to make the search results more
customizable. At the moment the user can only add and remove corresponding test
classes. A similar option to link and unlink individual test methods could be
useful.

Partially integrating some of the features of the TVPlugin directly into
Nautilus could also be beneficial. For example the TVPlugin's ability to split
the code editor in half could be used to show other things than methods and
corresponding tests side by side.

Another feature of the TVPlugin that could be integrated into Nautilus is its
test search. Nautilus' own test search is used to display a button besides each
method with which the corresponding test can be executed. With an enhanced test
search it would become possible to run all found tests instead of just one
through this button.

The test method templates that the TVPlugin creates to make adding new tests
easier could also be used by Nautilus to give each new test some default
content.

%%%%%%%%%%%%%%%%%%%%%%%%%%%%%%%%%%
%%%% Anleitung zu wissenschaftlichen Arbeiten %%%%%%%%%%%%%%%%%%%%%
%%%%%%%%%%%%%%%%%%%%%%%%%%%%%%%%%%
\chapter {Anleitung zu wissenschaftlichen Arbeiten}
	\label{cha:Anleitung zu wissenschaftlichen Arbeiten}
\section {User Guide}
	\label{sec:User Guide}
\subsection{What is the TestView Plugin?}
	\label{subsec:What's a TestView Plugin?}

Nautilus is the default system browser in Pharo 4.0.
\autoref{fig:NautilusWindow} shows how a Nautilus window
normally looks like.

\begin{figure}[H]
	\centering
	\includegraphics[scale=0.5]{screenshots/pharoClassHierarchy.png}
	\caption{A Nautilus Window}
	\label{fig:NautilusWindow}
\end{figure}

The TestView Plugin (or TVPlugin) is a Nautilus plugin to facilitate unit
testing. It provides quick ways to add new test methods and classes, find
existing tests and view tests and methods at the same time in a single Nautilus
window.

Let us take a look at the TVPlugin in \autoref{fig:TVPlugin overview} and
discuss the features that it provides.

\begin{figure}[H]
	\centering
	\includegraphics[scale=0.5]{screenshots/tvPluginOverView_edited.png}
	\caption{Nautilus with the TVPlugin}
	\label{fig:TVPlugin overview}
\end{figure}

The \textbf{``TestView''} button toggles the additional code panel to the right
of the original code panel on and off. All features described here require that
the plugin is turned on.

The TestView Plugin heavily relies on knowing which method is currently selected
in a Nautilus window.  All functions that the plugin
provides are relative to what you have selected in the Nautilus window.
Whenever you select a method in the Nautilus class hierarchy the plugin will
automatically search for corresponding tests.

In the \textbf{found tests drop down list} every test corresponding to the selected
method is shown. The first element in this list is special and will always be
there independently from which method you have selected in the Nautilus window.
Click this element to create a new test in a possibly new test class. How to do
this is explained in detail in \autoref{subsec:Creating a new test inside a new
test class}.  The remaining items in the list are all existing tests that
correspond to the method you selected in the Nautilus window. By clicking on
one of these the right code panel will display the selected test.

The \textbf{additional code panel} to the right of the original code panel will
always show the test that has been selected in the found tests drop down list. With
this it becomes possible to look at the implemented method and at the tests for
it inside of the same Nautilus window.

You can use the \textbf{``Link Class''} and \textbf{``Unlink Class'' buttons} if
the found tests for the method you selected in the Nautilus window are
incomplete or show methods that are not tests for what you selected. You can use
both these buttons to influence the automated test search that gets performed
whenever you select a different method in the Nautilus class hierarchy. Your changes
to the search results will be stored in your Pharo image.

With the ``Link Class'' button you can specify a test class that is not found by
the automated test search. The plugin will then redo the test search and include
the newly linked class as a possible source for tests on every search to the
class of the selected method.

With the ``Unlink Class'' button you can exclude classes from being searched for
tests. Like with the ``Link Class'' button this exclusion will only count for
the class of the method that you have currently selected in the Nautilus class
hierarchy. Detailed instructions on how to use these functionalities are found
in \autoref{subsec:Linking an existing test class} and \autoref{subsec:Unlinking
an existing test class}.

\subsection{Installation and Activation}
	\label{subsec:Installation and activation}

To install and activate the TVPlugin follow the steps listed below:

\begin{enumerate}
\item To download the necessary packages simply execute the following lines in a
Pharo workspace

\begin{code}
Gofer new
url: 'http://smalltalkhub.com/mc/DominicSina/TestView/main';
package: 'ConfigurationOfTestView';
load.
(Smalltalk at: #ConfigurationOfTestView) loadDevelopment.
NautilusPluginManager new openInWorld
\end{code}

Once this is finished the Nautilus Plugins Manager will open.

\item
In the window shown in \autoref{fig:pluginManager} you click on ``TVPlugin'' under
``Available plugin classes'' and then press on the ``Add'' button. Click ``Ok'' to
confirm.

\parbox{\linewidth}{\centering
\includegraphics[scale=0.5]{screenshots/pluginManager_edited.png}
\captionof{figure}{The Nautilus Plugin Manager}
\label{fig:pluginManager}
}

\item
When you open a Nautilus window from now it should look like in
\autoref{fig:activeTVP}. To verify if the plugin is activated check if the
highlighted row is displayed in the position that you selected. The plugin will
be shown there until you remove it again using the Nautilus Plugins Manager.

\parbox{\linewidth}{\centering
\includegraphics[scale=0.5]{screenshots/activatedPlugin_edited.png}
\captionof{figure}{TestView Plugin once it is activated}
\label{fig:activeTVP}
}

\end{enumerate}

\subsection{Adding a New Test Inside a New Test Class}
\label{subsec:Creating a new test inside a new test class}

So now that the plugin is set up after following the steps in
\autoref{subsec:Installation and activation} we can add a new test
for a method.  In this example it is assumed that you have a method to test
named ``doSomething'' inside of a class named ``MyClass'' and a package called
``MyPackage''.

\begin{enumerate}

\item Turn the TVPlugin on by clicking on the ``TestView'' button highlighted in
\autoref{fig:toggleButton}. Once this is done a second code panel will appear
besides the original code panel that shows a new test for your selected method.

		\parbox{\linewidth}{\centering
        \includegraphics[scale=0.5]{screenshots/toggleButton_edited.png}
        \captionof{figure}{Toggle the TVPlugin on}
        \label{fig:toggleButton}
    }

\item Make sure that in the Nautilus window you have selected the method for
which you want to add a test.

\item Now open the drop down list showing all the tests that have been found for your
method. Click on ``new Test'' as shown in \autoref{fig:dropListNewTest}. By
doing this you signal to the plugin that you want to add a test in a new test
class.

		\parbox{\linewidth}{\centering
        \includegraphics[scale=0.5]{screenshots/dropListNewTest_edited.png}
        \captionof{figure}{Signal that you want to save the test in a new test class}
\label{fig:dropListNewTest}
    }

\item Write your test in the right code panel. A template to start writing is
already provided there by the TVPlugin.

\item Make sure the right code panel is still selected and accept your new test
by pressing ctrl+s.

\item Since you previously selected ``new Test'' from the drop down list the plugin is
not sure in which test class this new test should be saved and will ask for
clarification. As shown in \autoref{fig:newTestClassName} a default name will
already be provided but you can write your own test class name. Click ``OK''
when you have entered a name.

		\parbox{\linewidth}{\centering
        \includegraphics[scale=0.5]{screenshots/nameTestClass_edited.png}
        \captionof{figure}{Name the new test class}
\label{fig:newTestClassName}
	}

\item Now you need to specify to which package this new
test class will be added. Similarly as before a default name will be provided
but you can enter your own. Click the ``OK'' button highlighted in
\autoref{fig:newTestPackage} to confirm the package name. This test package will
now be created if it did not exist previously.

  	\parbox{\linewidth}{\centering
        \includegraphics[scale=0.5]{screenshots/nameTestPackage_edited.png}
        \captionof{figure}{Name the new test Package}
\label{fig:newTestPackage}
	}

\item You will now see a pop-up similar to \autoref{fig:classConfirm} with a
class definition according to what you entered in the previous steps. You can
one last time change your mind and cancel the creation of the new class or enter
new class and package names. Once you have finished checking and if necessary
have made additional changes click on ``OK''. The new test will now be added to
the newly created test class and test package.

	\parbox{\linewidth}{\centering
        \includegraphics[scale=0.5]{screenshots/confirmNaming_edited.png}
        \captionof{figure}{Confirm the new test class}
\label{fig:classConfirm}
	}

\end{enumerate}

\subsection{Adding a New Test to an Existing Test Class}
	\label{subsec:Adding a new to an existing test class}

In this section we will take a look at how to add a new test inside of an
existing test class. It is assumed that the plugin is installed and activated.
If not follow the steps outlined in \autoref{subsec:Installation and
activation}.

\begin{enumerate}

\item First make sure that the plugin is toggled on. If it is the Nautilus
window has two code panels in the bottom. If it is not turn it on by clicking
the ``TestView'' button. See \autoref{fig:toggleButton} if you can not find the
button.

\item Make sure that  in the Nautilus window you have selected the method for
which you want to add a test.

\item Now expand the drop down list with the results and select any test that is
contained in the class where you want your new test to be. In this example this
will be ``MyClassTest'' so we select ``MyClassTest $>>$\#testDoSomething'' as
can be seen in \autoref{fig:dropListExistingClass}.

	\parbox{\linewidth}{\centering
        \includegraphics[scale=0.5]{screenshots/dropListExistingTestClass_edited.png}
        \captionof{figure}{Select any test in the desired test class}
\label{fig:dropListExistingClass}
	}

\item The test you selected will now appear in the right code panel. Just
write your test over it but make sure to give it a different name or else the
test you selected will be overwritten.

\item Make sure the right code panel is still selected and accept your new test
by pressing ctrl+s.

\item Now your new test will be added to the test class you selected previously.
You can verify this by opening the found tests drop down list again and checking if
your new test is there as can be seen in \autoref{fig:newTestCheck}.

	\parbox{\linewidth}{\centering
        \includegraphics[scale=0.5]{screenshots/VerifyNewTestIsThere_edited.png}
        \captionof{figure}{Check if your new test is shown here}
\label{fig:newTestCheck}
	}
\end{enumerate}

\subsection{Linking an Existing Test Class}
	\label{subsec:Linking an existing test class}

When the TVPlugin does not find some test methods then it might be because
their test class is not considered to be a test class for the
currently selected method. All test methods in this missing
test class will not be found and thus not be shown in the found tests drop
list. Here you find step by step instructions on how to add
a missing test class to the considered test classes.

\begin{enumerate}

\item Make sure that, in the Nautilus class hierarchy, you have
selected the method which is missing tests in the found tests drop down list.

\item Click the ``Link class'' button highlighted in \autoref{fig:linkClassButton}.

	\parbox{\linewidth}{\centering
        \includegraphics[scale=0.5]{screenshots/linkClassButton_edited.png}
        \captionof{figure}{The ``Link class'' button}
\label{fig:linkClassButton}
	}

\item The TVPlugin will now ask which class you want to link as a test class to
the class that is currently selected. Enter the name of the desired test class
and click ``OK'' as shown in \autoref{fig:linkClassName}.

	\parbox{\linewidth}{\centering
        \includegraphics[scale=0.5]{screenshots/linkClassName_edited.png}
        \captionof{figure}{Pop-up asking for the test class name}
\label{fig:linkClassName}
	}

\item The specified test class is now added to the test classes that will be
considered when searching for tests for the currently selected class. You can
verify if it now works by opening the found tests dropview.

\end{enumerate}

\subsection{Unlinking an Existing Test Class}
	\label{subsec:Unlinking an existing test class}

In case the TVPlugin shows you tests from a test class that you do not recognize
as a test class for the currently selected method you can remove this class from
consideration by using the ``Unlink class'' button.

\begin{enumerate}

\item First in the Nautilus class hierarchy select the method that does show
too many tests in the found tests drop down list.

\item Click the ``Unlink class' ' button highlighted in \autoref{fig:unlinkClassName}.

	\parbox{\linewidth}{\centering
        \includegraphics[scale=0.5]{screenshots/unlinkClassButton_edited.png}
        \captionof{figure}{The ``Unlink class'' button}
\label{fig:unlinkClassName}
	}

\item The TVPlugin will now ask which class you want to unlink as a test class
from the class that is currently selected. Enter the name of the desired test
class and click ``OK'' as shown in \autoref{fig:unlinkClassName}.

	\parbox{\linewidth}{\centering
        \includegraphics[scale=0.5]{screenshots/unlinkClassName_edited.png}
        \captionof{figure}{Pop-up asking for the test class name}
\label{fig:unlinkClassName}
	}

\item The specified test class is now removed from consideration. All tests
contained in this class will not be considered to be tests for the currently
selected class. You can verify if this now works by opening the found tests
drop down list. No test contained in the unlinked class should be shown there.

\end{enumerate}

\subsection{Troubleshooting}
	\label{subsec:Troubleshooting}

In this section possible problems with the TVPlugin are listed and possible
solutions to them outlined.

\subsubsection{Tests are Not Found by the TVPlugin}
\label{subsubsec:Tests are not found by the TVPlugin}

In case all tests from one or multiple tests classes are missing in the found
test drop down list you can add those test classes manually by following
\autoref{subsec:Linking an existing test class}. If you want to add individual
tests from a test class where already some tests are found then continue with
\autoref{subsubsec:Add and remove individual tests from the corresponding test
list to the selected method}.

First make sure that your test classes are subclasses of ``TestCase''. If they
are not the plugin will not consider them to be test classes. Simply go to the
definition of your test classes and if necessary change the first line to
``[YourClassName] subclass: \#TestCase''. Check again if the tests from your
test classes are found now by selecting the method for which some tests were not
found.

If this does not help you can force the plugin to recognize your classes as test
classes by linking the missing test classes to the currently selected class in
the Nautilus class hierarchy. To do this follow the steps in
\autoref{subsec:Linking an existing test class}. Recheck if your tests are
found. If this does not work either then your test methods themselves are written in a
way that the TVPlugin does not recognize. You will have to change them so that
they fulfill certain criteria. In this case also continue with
\autoref{subsubsec:Add and remove individual tests from the corresponding test
list to the selected method}.

\subsubsection{Tests That Do Not Test the Currently Selected Method Are Shown}
		\label{subsubsec:Tests that do not test the currently selected method are shown}

If the classes that contain these wrongly identified tests do not contain any
tests for the currently selected class then you can simply unlink these classes.
Follow the steps outlined in \autoref{subsec:Unlinking an existing test class}.
All the tests from these classes should now be gone from the found tests
drop down list.

If you want to remove only some tests from the found tests drop down list then
continue with \autoref{subsubsec:Add and remove individual tests from the
corresponding test list to the selected method}.

\subsubsection{Adding and Removing Specific Found Tests}
		\label{subsubsec:Add and remove individual tests from the corresponding test list to the selected method}

Both adding and removing individual test methods from the tests that the TVPlugin
finds are features that are not directly supported. It is still possible to make
these adjustments although not without changing your tests. The way the TVPlugin identifies
individual tests as corresponding to the selected method has two components. The
first one is solely based on the name of your test and the name of the method
that it is supposed to test. The second one is if the test calls a method with
the same name as the method that it supposedly tests.

The name based criterion is a check if your test contains the full name of your
method under test with at least one additional time the substring ``test''.
Examples of what is recognized as a test and what not based on this criterion
are shown in \autoref{tab:name criterion examples}. This check is not case
sensitive. The ``:'' character from methods names with parameters will
be ignored when looking for a test.

\begin{table}[h!]
	\centering
	\begin{tabular}{| c | c | c |}
     	\hline
     	\emph{Original method name} & \emph{Test names} & \emph{Not test names} \\ \hline
    	 addDays: & testAddDays & addDays, testAddDay \\
    	 asFloat & testFractionAsFloat, testFractionAsFloat2 & testFloatAsFraction \\
	print24:on: & testPrint24On, testPrint24OnWithPM & testPrint24withNanos \\
	 attest & attestTest, testAttest & attest \\
	\hline
   	\end{tabular}
	\captionof{table}{Test naming criterion}
	\label{tab:name criterion examples}
\end{table}

The second criterion is if a method call is done inside of the possible test
method to a method with the same name as the in the Nautilus class hierarchy
selected method. This time the ``:'' characters are not removed from the check. If the
supposed test method of a method named ``do:on:'' does call ``doOn'' then it would
still not count as a test.

A test is recognized as such if either or both of these criteria are fulfilled.
Conversely if neither is then the test is not considered a test for the selected
method.  Adding a specific method to the tests found for a certain method can
thus be done in two ways. Either make the name of the test contain the full name
of the method in addition to ``test'' or call the method you want to test directly
in the test. To remove a method from the found tests you have to make sure that
the test name does not contain ``test'' and the full method name, and that inside
of the test never a method with the same name as the supposed method under test
is called.

\bibliography{scg,thesis}
	\bibliographystyle{plain}

\end{document}
